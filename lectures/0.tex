\chapter{Организационная лекция}

Лектор: Алексей Сергеевич Забашта

\section{План курса}
\begin{enumerate}
    \item Гиперпараметры
    \begin{itemize}
        \item Настройка гиперпараметров
    \end{itemize}
    \item Глубокое обучение
    \begin{itemize}
        \item Автоматическое дифференцирование
        \item Базовые архитектуры
        \item Свёрточные и рекуррентные сети
        \item Анализ не табличных данных
    \end{itemize}
    \item Задачи обучения без учителя
    \begin{itemize}
        \item Кластеризация
        \item Выделение признаков
        \item Генерация данных
    \end{itemize}
\end{enumerate}

\section{Система оценки}

Для получения оценки есть следующие пути (не взаимозаменяемые):
\begin{itemize}
    \item Практика --- сдача задач и лабораторных работ
    \item Теория --- написание контрольной \textbf{или} сдача экзамена
    \item Бонусные баллы
\end{itemize}

Финальная оценка вычисляется по следующей формуле:
\[
    FinalScore = \min\left(\sqrt{T\cdot P} + B', 100\right)
\]
где
\begin{itemize}
    \item $T\in[0; 120]$ --- баллы за теоретическую часть;
    \item $P\in[0; 120]$ --- баллы за практическую часть;
    \item $B'\in[0; 30)$ --- скорректированные бонусные баллы, которые получаются из обычных мягко ограничивающей функцией:
    \[
        B' = \dfrac{30B}{30 + B}.
    \]
\end{itemize}

Баллы за теорию можно получить либо на контрольной работе, на которой можно получить от 0 до 60 баллов, либо на экзамене:
\begin{itemize}
    \item Экзамен не суммируется с контрольной;
    \item В случае попытки сдачи баллы за теорию будут равны баллу за экзамен;
    \item Состоит из теормина (60 баллов) и ответов на вопросы билета (ещё 60 баллов).
\end{itemize}

Баллы за практику можно получать за лабораторные работы и задачи. Первые нужно защищать на практиках и по ним есть мягкий дедлайн: за них можно получить $K\cdot\left(0,6 + \frac{0,4}{1 + w}\right)$ баллов, где $w$ - это число недель после дедлайна. Задачи сдаются на \texttt{https://codeforces.com/}, их не нужно защищать и по ним строгий дедлайн. Распределение баллов между лабораторными работами и задачами - 70/50 (суммарно можно набрать 120 баллов).

Бонусные баллы можно получить за:
\begin{itemize}
    \item написание конспектов
    \item написание визуализаторов
    \item улучшение курса
\end{itemize}
